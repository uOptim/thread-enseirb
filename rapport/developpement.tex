La première étape a été de récupérer le squelette du code de la bibliothèque de thread et de coder les fonctions nécessaires à son fonctionnement. Cette partie explique les choix qui ont été faits, ainsi que les problèmes rencontrés et la façon dont nous les avons résolus.

\subsection{Choix de programmations}

L'initialisation de la bibliothèque est implicite et ne nécessite pas d'être
faite manuellement par le code client. En effet, des fonctions \verb!__init()!
(resp. \verb!__destroy()!) avec l'attribut \verb!__attribute__((constructor))!
(resp. \verb!__attribute__((destructor))!) fourni par gcc permettent de
réaliser les initialisations (resp. le nettoyage) nécessaires au bon
fonctionnement.

Les listes de \verb!sys/queue.h! empruntées à BSD ont été retenues pour leur
simplicité d'utilisation et leur "légèreté". Les files et listes doublement
chaînées permettent des ajouts et suppressions en temps constant et ces
oppérations peuvent être réalisées pendant leur parcours
(\verb!*_FOREACH_SAFE!).


\subsection{Version de base sans objectif avancé}

\paragraph{} La version de base garde une liste de threads utilisateur à exécuter. Ceux-ci sont représentés par des contextes qui sont exécutés l'un après l'autre lors d'un appel à \verb!thread_yield()!. Un pointeur vers le thread courrant et un autre vers son succésseur sont mis à jour à chaque changement de contexte pour fournir un ordonnancement rudimentaire qui donne son tour à chaque thread.

\paragraph{} Le cas des threads ne faisant pas appel à \verb!thread_exit()! a été traîté à l'aide d'une fonction auxiliaire qui l'invoque explicitement après appel à la fonction donnée par l'utilisateur (voir fonction \verb!_run()!). Lorsqu'un thread a terminé (après un appel à \verb!thread_exit()!), il est retiré de la liste de threads à exécuter, sa valeur de retour est sauvegardée et il est marqué comme étant terminé. Ses ressources ne sont libérées que lorsqu'il est la cible d'un \verb!thread_join()!.


\subsection{Définition d'une structure}
Comment, quels champs 

\subsection{Fonctionnement général du programme}
\subsubsection{Variables globales}
\subsubsection{Thread yield}
\subsubsection{Thread create}
\subsubsection{etc ...}
