La première étape a été de récupérer le squelette du code de la bibliothèque de
thread et de coder les fonctions nécessaires à son fonctionnement. Cette partie
explique les choix qui ont été faits, ainsi que les problèmes rencontrés et la
façon dont nous les avons résolus.

\subsection{Choix de programmations}

L'initialisation de la bibliothèque est implicite et ne nécessite pas d'être
faite manuellement par le code client. En effet, des fonctions \verb!__init()!
(resp. \verb!__destroy()!) avec l'attribut \verb!__attribute__((constructor))!
(resp. \verb!__attribute__((destructor))!) fourni par gcc permettent de
réaliser les initialisations (resp. le nettoyage) nécessaires au bon
fonctionnement.

Les listes de \verb!sys/queue.h! empruntées à BSD ont été retenues pour leur
simplicité d'utilisation et leur "légèreté". Les files et listes doublement
chaînées permettent des ajouts et suppressions en temps constant et ces
oppérations peuvent être réalisées pendant leur parcours
(\verb!*_FOREACH_SAFE!).


\subsection{Le cas du thread principal}

La gestion du thread principal nécessite des opérations différentes des autres
threads notamment lors de la libération de ressource ou de l'ordonnancement. En
effet, les tests qui nous sont fournis demandent à ce que le thread principal
fasse des \verb!yield! vers des fils et un fils doit faire des \verb!yield!
vers le thread principal (cf. \verb!02-switch.c!). De plus, le thread principal
doit pouvoir être utilisé dans la fonction \verb!thread_join()! tout en
permettant aux autres threads de continuer à utiliser \verb!yield!.
