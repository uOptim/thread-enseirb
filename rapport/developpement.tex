La première étape a été de récupérer le squelette du code de la bibliothèque de thread et de coder les fonctions nécessaire à son fonctionnement. Cette partie explique les choix qui ont été faits, ainsi que les problèmes rencontrés et la façon dont nous les avons résolus.
\subsection{Choix de programmations}
- initalisation \\
- choix de la liste \\
- execution du programme d'exemple --> Valgrind\\
- etc

\subsection{Le cas du thread principal}

Les tests qui nous sont fourni demandent à ce que le thread principal fasse
des \verb!yield! vers des fils et un fils doit faire des \verb!yield! vers le
thread principal. De plus, le thread principal doit pouvoir être utilisé dans
la fonction \verb!thread_join()! tout en permettant aux autres threads de
continuer à utiliser \verb!yield!.