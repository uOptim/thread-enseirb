Ce projet s'inscrit dans la continuité du cours et des TDs du module de Systèmes d'exploitation. Le but de ce projet et de mettre en place une bibliothèque de gestion des threads en espace utilisateur. Celle-ci reprend le comportement de la bibliothèque exisante $pthread$. Contrairement à $pthread$, la bibliothèque que nous devons créer s'exécute en un seul thread noyau, ce qui permet plus de flexibilité au terme de programmation. \\
\indent Dans un premier temps, nous avons défini une structure représentant un thread. Nous avons ensuite implémenté les fonctions définies dans le fichier $thread.h$ donné dans l'énoncé. Dans un deuxième temps, nous avons testé cette bibliothèque sur les exemples fournis et sur d'autres, que nous avons impléméntés (tris de tableaux, calcul de la somme d'un tableau). Enfin, nous avons commencé à réaliser des fonctions avancées telles que la préemption, les threads noyaux et la fonction cancel. 

