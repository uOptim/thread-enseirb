Dans cette partie, afin d'alléger les phrases, une tâche est synonyme de thread utilisateur (que l'on doit distinguer des threads noyaux).

\subsubsection{Aperçu du fonctionnement}

Afin d'utiliser tous les processeurs disponibles, un certain nombre (fixé à la compilation) de threads noyaux sont créés lors de l'initialisation de la bibliothèque. Ce nombre inclut le thread principal qui n'est pas distinguable des threads créés du point de vue de l'utilisateur. Les threads alors créés entrent dans une boucle infinie dans laquelle ils vont consommer des tâches à réaliser placés dans une file.

Ces tâches sont celles créées par l'utilisateur via la fonction \verb!thread_create()! de la bibliothèque. Elles sont définies par un contexte et quelques attributs sur leur état.


\subsubsection{Creation de threads noyaux}

\paragraph{Avec clone()}
Nous avons tout d'abord utilisé l'appel système \verb!clone()! avec notamment les flags \verb!CLONE_VM! (partage de la mémoire), \verb!CLONE_THREADS! (processus créés dans un même groupe de threads), etc. Cette solution a bien abouti mais nous a permis de nous rendre compte de plusieurs problèmes.

Le premier est le stockage de données propres au thread qui sont utiles pour les distinguer et rendre l'appel à \verb!thread_self()! efficace. L'idée était alors de conserver un pointeur global vers le thread utilisateur courant qui lui permettrait, lorsqu'il est exécuté sur un thread noyau, d'avoir un accès immédiat vers la structure qui le représente. Une solution (non portable) proposée par gcc est de déclarer des variables avec l'attribut \verb!__thread! mais nous ne sommes pas parvenu à obtenir le comportement souhaité car la valeur du pointeur semblait partagée entre tous les processus malgré tout. Nous avons donc changé d'approche et avons résolu notre problème en créant un tableau associant un l'identifiant d'un thread noyau (obtenu via \verb!gettid()!) à un thread utilisateur. Cette solution nécessite donc le parcours d'un tableau de taille égale au nombre de threads noyaux à chaque fois que l'on souhaite identifier quel thread utilisateur est exécuté.

Le second problème provient de glibc. En effet, lorsque l'on dit que glibc est
'thread-safe', il faut en fait comprendre 'pthread-safe'
\footnote{\url{http://sourceware.org/ml/libc-alpha/2006-01/msg00086.html}}
\footnote{\url{http://osdir.com/ml/lib.glibc.bugs/2003-01/msg00012.html}}.
Nous nous en sommes rendu compte lors d'appels intensifs vers \verb!malloc!
pour la création de threads utilisateur dans les tests comme fibonacci : nous
obtenions des corruptions du tas laissant penser que malloc n'était pas
thread-safe!  L'explication est simple : l'ensemble de glibc dépend énormément
du fonctionnement de pthread pour les applications multi-threadées et fait des
ajustements pour avoir un comportement correct dans ce type de programme. En
voulant se passer de pthread au profit d'une implémentation plus bas niveau
grâce à \verb!clone()! avec \verb!CLONE_VM!, nous contournons ces ajustements
et l'utilisation de la bibliothèque standard n'est alors plus sûre. La
corruption du tas fût corrigée avec un simple mutex placé avant chaque
\verb!malloc()! ou \verb!free()!, confirmant notre observation. Ceci est lourd
de conséquence car la même précaution doit alors être prise par l'utilisateur
et cela pour toutes les fonctions de la bibliothèque standard pouvant souffrir
d'appels concurrents!


\paragraph{Avec pthread\_create()}
Afin de supprimer la limitation précédente, nous avons remplacé l'appel à \verb!clone()! par un appel à \verb!pthread_create()!. Les mutex avant l'allocation et la libération de mémoire ont alors pu être supprimées sans faire réaparaitre de corruption et nous avons pu utiliser des \verb!pthread_key! pour stocker un pointeur vers la tâche courante et s'affranchir du parcours de tableau précédent.


\subsubsection{Ordonnancement des tâches}

L'ordonnancement de plusieurs tâches sur l'ensemble des threads noyaux est réalisé grâce aux techniques suivantes :

\paragraph{File de tâches} Les threads utilisateur sont stockés dans une file. Un mutex et un sémaphore y sont associés pour la concurrence et le signalement de nouvelles tâches enfilées respectivement.

\paragraph{Contexte d'un thread noyau} À l'initialisation, on distingue deux cas pour les threads noyaux. Le thread principal (qui est exécuté depuis le début du programme) n'est pas altéré: il continue l'exécution normale du programme dans un premier temps tandis que les threads créés attendent les premières tâches à exécuter dans une boucle infinie qui va consommer des tâches placées dans la file (fonction \verb!_clone_func()!).

\paragraph{Exécution des threads utilisateur} Pour chaque thread noyau, le thread utilisateur exécuté est responsable de l'exécution du thread utilisateur suivant. 

En fonction de la quantité de threads noyaux et de tâches, trois cas peuvent se présenter :
\begin{itemize}
	\item Si la file de tâches est vide et que le thread utilisateur courant est terminé, alors on repasse dans le contexte thread noyau pour attendre de nouvelles tâches sauf s'il l'on détecte que nous sommes le tout dernier thread exécuté auquel cas le programme termine. Dans le cas où ce thread noyau est le thread principal, on lance la fonction d'attente sur la file \verb!_clone_func()!.
	\item Sinon si la file est vide mais que le thread courant n'est pas fini, alors ne rien faire et continuer l'exécution.
	\item Sinon, la file n'est pas vide donc appeler une fonction de changement de contexte \verb!_magicswap()! détaillée plus loin.
\end{itemize}

\paragraph{Changements de contextes avec \_magicswap()} Afin d'éviter qu'une même tâche soit manipulée par plusieurs threads noyaux à la fois, on associe à chacune d'elles un mutex qui doit être pris par le un thread noyau avant toute opération. Un thread noyau souhaitant exécuter une tâche doit avoir pris le verrou et le conserver pendant toute l'exécution, y compris avant et après le \verb!swapcontext!.

Avec ce schéma et s'il l'on souhaite ne pas repasser au contexte du thread noyau pour réaliser la libération de l'ancienne tâche et le verrouillage d'une nouvelle, il faut pouvoir indiquer à une tâche sa tâche appelante afin de réaliser la libération.

Une tâche en appelant une autre doit donc préparer la tâche suivante en vérouillant son mutex, en lui indiquant qui l'a appelé, en changeant le pointeur de tâche courante du thread noyau et enfin en réalisant le changement de contexte. Une tâche venant d'être exécutée doit quant à elle récupérer son identité grâce au pointeur de tâche courante du thread noyau, libérer la tâche appelante ainsi que la replacer dans la file si elle n'est pas terminée et enfin continuer son exécution normale.
