Nous avons utilisé la fonction $ualarm$ pour envoyer un signal d'alarme $SIGALRM$ à intervalles réguliers. Le gestionnaire de signaux associé réalise les opérations de préemption.

Dans un premier temps, il passait la main du contexte courant à un contexte spécialement dédié à l'ordonnancement. Ce contexte ordonnanceur passait la main au contexte suivant de la liste $ready$.  Nous avons cependant rencontré des difficultés à implémenter la préemption de cette manière (incohérences lors des changements de contexte). De plus, cette solution ne semblait pas optimale, en raison d'un nombre élevé de changements de contextes, à la fois coûteux et peu utiles. Par conséquent, nous avons finalement appelé directement la fonction \verb!thread_yield()! dans le handler. 

Nos tests (en particulier \verb!55-increment!) montrent que les signaux sont bien pris en compte. En revanche des problèmes persistent et des erreurs de segmentation surviennent lors d'une exécution prolongée.

La préemption n'a été explorée qu'à partir de la version tournant sur un seul thread noyau. Si notre méthode avait abouti, nous aurions implémenté ce procédé sur la version utilisant plusieurs threads noyaux qui aurait alors nécessité des ajustements. Par exemple, la propagation du signal SIGALRM sur tous les threads (sachant qu'un seul thread ne reçoit le signal délivré par le noyau) grâce à un second signal.
