\documentclass[a4paper,12pt]{article}
\usepackage[utf8]{inputenc}
\usepackage[T1]{fontenc}
\usepackage[french]{babel}
\usepackage{graphicx}
\usepackage{alltt}
\usepackage{listings}
\usepackage{moreverb}
\usepackage[colorlinks=true,linkcolor=black,urlcolor=blue]{hyperref}
\usepackage{tikz}

\usetikzlibrary{mindmap,trees}
\addtolength{\hoffset}{-1cm}
\addtolength{\textwidth}{2cm}

\definecolor{rouge}{HTML}{993333} 

\title{
	\normalsize{ENSEIRB-MATMECA \\ 
	2012 - 2013 \\
	2ème année informatique} \\
	\vspace{15mm}
	\Huge{\textcolor{blue}{Projet de Système d'Exploitation }}
} 
\author{ BENOIT Amélie \\ BRISSET Clément \\ LANDEIRO DOS REIS Virgile \\ RABENJAMINA Samantha \\ RUELLE BENOIT}

\date{
	\normalsize{Enseignant encadrant : François TESSIER}
}

\begin{document}

\maketitle

\clearpage

\section{Introduction et mise en route}

- But et objectifs du projet \\

Réponse aux questions, changement de contexte etc.
En quoi ça nous a servi pour la suite

\section{Développement de la bibliothèque}
La première étape a été de récupérer le squelette du code de la bibliothèque de thread et de coder les fonctions nécessaire à son fonctionnement. Cette partie explique les choix qui ont été faits, ainsi que les problèmes rencontrés et la façon dont nous les avons résolus.
\subsection{Choix de programmations}
- initalisation \\
- choix de la liste \\
- execution du programme d'exemple --> Valgrind\\
- etc

\subsection{Le cas du thread principal}

Les tests qui nous sont fourni demandent à ce que le thread principal fasse
des \verb!yield! vers des fils et un fils doit faire des \verb!yield! vers le
thread principal. De plus, le thread principal doit pouvoir être utilisé dans
la fonction \verb!thread_join()! tout en permettant aux autres threads de
continuer à utiliser \verb!yield!.

\section{Tests de robustesse et de performance}
\subsection{Les tests de base}
\subsection{Tests implémentés}
\subsubsection{Somme des éléments d'un tableau}
\subsubsection{Les tris tableau}

\clearpage
\section{Réflexion sur les objectifs avancés}
- Préemption : créer un thread ( !!! différent de notre bibliothèque (pthread) , sinon on ne pourra jamais ordonnancer) en parallèle du main qui ferait des getTimeofDay et lancerait une fonction thead\_schedule toutes les X millisecondes. Cette fonction passerait la main au thread suivant dans la liste.\\

- Priorité -- une fois qu'on a géré la préemption ...
\clearpage
\section{Conclusion}


\end{document}
