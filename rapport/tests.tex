\subsection{Les tests de base}
\subsection{Tests implémentés}

\subsubsection{Les tris de tableaux}
Les algorithmes de tri de tableaux $tri rapide$ et $tri fusion$ ont été implémentées. Dans le cas du $tri rapide$, un thread est créé traîter chaque partie du tableau de part et d'autre du pivot. Pour le $tri fusion$, chaque fois qu'un tableau est divisé en deux, chaque partie est traitée par un nouveau thread.\\

%%%% resultats tests + include graphics %%%%
%\begin{figure}[H]
%\includegraphics[scale=0.4]{pouetpouet.png}
%\caption{Temps d'exécution des tests d'un tri rapide}
%\end{figure} 


\subsubsection{Somme des éléments d'un tableau}
Pour calculer la somme des éléments d'un tableau, nous utilisons le même principe que pour le tri fusion, à savoir diviser le tableau en deux et attribuer les deux parties à des threads distincts. Une fois que l'on obtient des tableaux d'une case, leur valeur est retournée et le thread parent calcule la somme des deux valeurs récupérées.

 %%%% resultats tests + include graphics %%%%
%\begin{figure}[H]
%\includegraphics[scale=0.4]{pouetpouet.png}
%\caption{Temps d'exécution des tests d'un tri rapide}
%\end{figure} 
